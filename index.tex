% Options for packages loaded elsewhere
\PassOptionsToPackage{unicode}{hyperref}
\PassOptionsToPackage{hyphens}{url}
\PassOptionsToPackage{dvipsnames,svgnames,x11names}{xcolor}
%
\documentclass[
]{article}
\usepackage{amsmath,amssymb}
\usepackage{lmodern}
\usepackage{iftex}
\ifPDFTeX
  \usepackage[T1]{fontenc}
  \usepackage[utf8]{inputenc}
  \usepackage{textcomp} % provide euro and other symbols
\else % if luatex or xetex
  \usepackage{unicode-math}
  \defaultfontfeatures{Scale=MatchLowercase}
  \defaultfontfeatures[\rmfamily]{Ligatures=TeX,Scale=1}
\fi
% Use upquote if available, for straight quotes in verbatim environments
\IfFileExists{upquote.sty}{\usepackage{upquote}}{}
\IfFileExists{microtype.sty}{% use microtype if available
  \usepackage[]{microtype}
  \UseMicrotypeSet[protrusion]{basicmath} % disable protrusion for tt fonts
}{}
\makeatletter
\@ifundefined{KOMAClassName}{% if non-KOMA class
  \IfFileExists{parskip.sty}{%
    \usepackage{parskip}
  }{% else
    \setlength{\parindent}{0pt}
    \setlength{\parskip}{6pt plus 2pt minus 1pt}}
}{% if KOMA class
  \KOMAoptions{parskip=half}}
\makeatother
\usepackage{xcolor}
\IfFileExists{xurl.sty}{\usepackage{xurl}}{} % add URL line breaks if available
\IfFileExists{bookmark.sty}{\usepackage{bookmark}}{\usepackage{hyperref}}
\hypersetup{
  pdftitle={SCHEDULE},
  colorlinks=true,
  linkcolor={Maroon},
  filecolor={Maroon},
  citecolor={Blue},
  urlcolor={blue},
  pdfcreator={LaTeX via pandoc}}
\urlstyle{same} % disable monospaced font for URLs
\usepackage[margin=1in]{geometry}
\usepackage{longtable,booktabs,array}
\usepackage{calc} % for calculating minipage widths
% Correct order of tables after \paragraph or \subparagraph
\usepackage{etoolbox}
\makeatletter
\patchcmd\longtable{\par}{\if@noskipsec\mbox{}\fi\par}{}{}
\makeatother
% Allow footnotes in longtable head/foot
\IfFileExists{footnotehyper.sty}{\usepackage{footnotehyper}}{\usepackage{footnote}}
\makesavenoteenv{longtable}
\usepackage{graphicx}
\makeatletter
\def\maxwidth{\ifdim\Gin@nat@width>\linewidth\linewidth\else\Gin@nat@width\fi}
\def\maxheight{\ifdim\Gin@nat@height>\textheight\textheight\else\Gin@nat@height\fi}
\makeatother
% Scale images if necessary, so that they will not overflow the page
% margins by default, and it is still possible to overwrite the defaults
% using explicit options in \includegraphics[width, height, ...]{}
\setkeys{Gin}{width=\maxwidth,height=\maxheight,keepaspectratio}
% Set default figure placement to htbp
\makeatletter
\def\fps@figure{htbp}
\makeatother
\setlength{\emergencystretch}{3em} % prevent overfull lines
\providecommand{\tightlist}{%
  \setlength{\itemsep}{0pt}\setlength{\parskip}{0pt}}
\setcounter{secnumdepth}{-\maxdimen} % remove section numbering
\usepackage[T2A]{fontenc}
\usepackage[utf8]{inputenc}
\usepackage[russian]{babel}
\usepackage{hyperref}
\hypersetup{colorlinks = false,pdfborder={1 1 1}}
\ifLuaTeX
  \usepackage{selnolig}  % disable illegal ligatures
\fi

\title{SCHEDULE}
\usepackage{etoolbox}
\makeatletter
\providecommand{\subtitle}[1]{% add subtitle to \maketitle
  \apptocmd{\@title}{\par {\large #1 \par}}{}{}
}
\makeatother
\subtitle{Nonlinear analysis and extremal problems (NLA-2022)}
\author{}
\date{\vspace{-2.5em}}

\begin{document}
\maketitle

\vspace{-25pt}

\hypertarget{zoom-link}{%
\subsubsection{Zoom link}\label{zoom-link}}

Meeting ID:

Passcode:

\hypertarget{timetable}{%
\section{Timetable}\label{timetable}}

\emph{UTC+8, Irkutsk local time}

\hypertarget{july-15-friday}{%
\subsubsection{July 15, Friday}\label{july-15-friday}}

\begin{longtable}[]{@{}
  >{\raggedright\arraybackslash}p{(\columnwidth - 2\tabcolsep) * \real{0.3125}}
  >{\raggedright\arraybackslash}p{(\columnwidth - 2\tabcolsep) * \real{0.6875}}@{}}
\toprule
\begin{minipage}[b]{\linewidth}\raggedright
\protect\includegraphics[height=0.7em]{/tmp/Rtmp15D9Ng/file12e4336d5fa7b.pdf}
Time
\end{minipage} & \begin{minipage}[b]{\linewidth}\raggedright
Activity
\end{minipage} \\
\midrule
\endhead
9:00 - 11:00 & Registration \\
11:00 - 11:10 & Opening ceremony \\
11:10 - 12:00 & \textbf{A. Kruger} (Lecture 1) \\
12:00 - 14:00 & Lunch \\
14:00 - 14:50 &
\protect\includegraphics[height=0.7em]{/tmp/Rtmp15D9Ng/file12e434155c59.pdf}
\textbf{G. Magaril-Il'yaev} (Lecture 1) \\
14:50 - 15:10 & Coffee break \\
15:10 - 16:00 &
\protect\includegraphics[height=0.7em]{/tmp/Rtmp15D9Ng/file12e435420f778.pdf}
\textbf{G. Magaril-Il'yaev} (Lecture 2) \\
16:00 - 16:30 & Coffee break \\
16:30 - 18:30 & \protect\hyperlink{se}{Sessions} \\
18:30 & Welcome reception \\
\bottomrule
\end{longtable}

\hypertarget{july-16-saturday}{%
\subsubsection{July 16, Saturday}\label{july-16-saturday}}

\begin{longtable}[]{@{}
  >{\raggedright\arraybackslash}p{(\columnwidth - 2\tabcolsep) * \real{0.3111}}
  >{\raggedright\arraybackslash}p{(\columnwidth - 2\tabcolsep) * \real{0.6889}}@{}}
\toprule
\begin{minipage}[b]{\linewidth}\raggedright
\protect\includegraphics[height=0.7em]{/tmp/Rtmp15D9Ng/file12e434fc8134d.pdf}
Time
\end{minipage} & \begin{minipage}[b]{\linewidth}\raggedright
Activity
\end{minipage} \\
\midrule
\endhead
10:00 - 10:50 & \textbf{V. Bogachev} (Lecture 1) \\
10:50 - 11:10 & Coffee break \\
11:10 - 12:00 & \textbf{S. Shaposhnikov} (Lecture 1) \\
12:00 - 14:00 & Lunch \\
14:00 - 14:50 & \textbf{A. Kruger} (Lecture 2) \\
14:50 - 15:10 & Coffee break \\
15:10 - 16:00 & \textbf{A. Kruger} (Lecture 3) \\
16:00 - 16:30 & Coffee break \\
16:30 - 18:30 & \protect\hyperlink{se}{Sessions} \\
\bottomrule
\end{longtable}

\hypertarget{july-17-sunday}{%
\subsubsection{July 17, Sunday}\label{july-17-sunday}}

\begin{longtable}[]{@{}
  >{\raggedright\arraybackslash}p{(\columnwidth - 2\tabcolsep) * \real{0.3111}}
  >{\raggedright\arraybackslash}p{(\columnwidth - 2\tabcolsep) * \real{0.6889}}@{}}
\toprule
\begin{minipage}[b]{\linewidth}\raggedright
\protect\includegraphics[height=0.7em]{/tmp/Rtmp15D9Ng/file12e433441997f.pdf}
Time
\end{minipage} & \begin{minipage}[b]{\linewidth}\raggedright
Activity
\end{minipage} \\
\midrule
\endhead
10:00 - 10:50 & \textbf{V. Bogachev} (Lecture 2) \\
10:50 - 11:10 & Coffee break \\
11:10 - 12:00 & \textbf{V. Bogachev} (Lecture 3) \\
12:00 - 14:00 & Lunch \\
14:00 - 14:50 & \textbf{S. Shaposhnikov} (Lecture 2) \\
14:50 - 15:10 & Coffee break \\
15:10 - 16:00 & \textbf{S. Shaposhnikov} (Lecture 3) \\
16:00 - 16:30 & Coffee break \\
16:30 - 18:30 & \protect\hyperlink{se}{Sessions} \\
\bottomrule
\end{longtable}

\hypertarget{july-18-monday}{%
\subsubsection{July 18, Monday}\label{july-18-monday}}

Trip to Baikal

\hypertarget{july-19-tuesday}{%
\subsubsection{July 19, Tuesday}\label{july-19-tuesday}}

\begin{longtable}[]{@{}
  >{\raggedright\arraybackslash}p{(\columnwidth - 2\tabcolsep) * \real{0.3125}}
  >{\raggedright\arraybackslash}p{(\columnwidth - 2\tabcolsep) * \real{0.6875}}@{}}
\toprule
\begin{minipage}[b]{\linewidth}\raggedright
\protect\includegraphics[height=0.7em]{/tmp/Rtmp15D9Ng/file12e43164d1bfd.pdf}
Time
\end{minipage} & \begin{minipage}[b]{\linewidth}\raggedright
Activity
\end{minipage} \\
\midrule
\endhead
9:00 - 9:50 &
\protect\includegraphics[height=0.7em]{/tmp/Rtmp15D9Ng/file12e435b634f8a.pdf}
\textbf{B. Mordukhovich} (Lecture 1) \\
9:50 - 10:10 & Coffee break \\
10:10 - 11:00 &
\protect\includegraphics[height=0.7em]{/tmp/Rtmp15D9Ng/file12e436dda1f9c.pdf}
\textbf{B. Mordukhovich} (Lecture 2) \\
11:00 - 14:00 & Lunch \\
14:00 - 14:50 &
\protect\includegraphics[height=0.7em]{/tmp/Rtmp15D9Ng/file12e43615b213e.pdf}
\textbf{L. Lokutsievskiy} (Lecture 1) \\
14:50 - 15:10 & Coffee break \\
15:10 - 16:00 &
\protect\includegraphics[height=0.7em]{/tmp/Rtmp15D9Ng/file12e432d0e04a5.pdf}
\textbf{L. Lokutsievskiy} (Lecture 2) \\
16:00 - 16:30 & Coffee break \\
16:30 - 18:50 & \protect\hyperlink{se}{Sessions} \\
\bottomrule
\end{longtable}

\hypertarget{july-20-wednesday}{%
\subsubsection{July 20, Wednesday}\label{july-20-wednesday}}

\begin{longtable}[]{@{}
  >{\raggedright\arraybackslash}p{(\columnwidth - 2\tabcolsep) * \real{0.3125}}
  >{\raggedright\arraybackslash}p{(\columnwidth - 2\tabcolsep) * \real{0.6875}}@{}}
\toprule
\begin{minipage}[b]{\linewidth}\raggedright
\protect\includegraphics[height=0.7em]{/tmp/Rtmp15D9Ng/file12e43200b5a9c.pdf}
Time
\end{minipage} & \begin{minipage}[b]{\linewidth}\raggedright
Activity
\end{minipage} \\
\midrule
\endhead
9:00 - 9:50 &
\protect\includegraphics[height=0.7em]{/tmp/Rtmp15D9Ng/file12e434e87b2a4.pdf}
\textbf{B. Mordukhovich} (Lecture 3) \\
9:50 - 10:10 & Coffee break \\
10:10 - 12:10 & \protect\hyperlink{se}{Session 2} \\
12:10 - 14:00 & Lunch \\
14:00 - 14:50 &
\protect\includegraphics[height=0.7em]{/tmp/Rtmp15D9Ng/file12e4331d04b36.pdf}
\textbf{L. Lokutsievskiy} (Lecture 3) \\
14:50 - 15:10 & Coffee break \\
15:10 - 17:30 & \protect\hyperlink{se}{Session 3} \\
17:30 - 17:40 & Closing ceremony \\
\bottomrule
\end{longtable}

\hypertarget{mini-courses}{%
\section{Mini courses}\label{mini-courses}}

\begin{enumerate}
\def\labelenumi{\arabic{enumi}.}
\tightlist
\item
  \textbf{Alexander Kruger}. Variational analysis and optimization
  theory: selected topics.
\item
  \textbf{Georgii Magaril-Il'yaev}. Controllability and optimality.
\item
  \textbf{Vladimir Bogachev}. Geometry and topology of the spaces of
  measures.
\item
  \textbf{Stanislav Shaposhnikov}. Nonlinear Fokker-Planck-Kolmogorov
  equations.
\item
  \textbf{Boris Mordukhovich}. Optimal control of sweeping processes.
\item
  \textbf{Lev Lokutsievskiy}. Introduction to sub-Riemannian and
  sub-Finsler geometries from the optimal control viewpoint. \newpage
\end{enumerate}

\hypertarget{se}{%
\section{Sessions}\label{se}}

\begin{longtable}[]{@{}
  >{\raggedright\arraybackslash}p{(\columnwidth - 6\tabcolsep) * \real{0.0938}}
  >{\raggedright\arraybackslash}p{(\columnwidth - 6\tabcolsep) * \real{0.2760}}
  >{\raggedright\arraybackslash}p{(\columnwidth - 6\tabcolsep) * \real{0.3490}}
  >{\raggedright\arraybackslash}p{(\columnwidth - 6\tabcolsep) * \real{0.2812}}@{}}
\toprule
\begin{minipage}[b]{\linewidth}\raggedright
Day
\end{minipage} & \begin{minipage}[b]{\linewidth}\raggedright
Session 1 (Room A)
\end{minipage} & \begin{minipage}[b]{\linewidth}\raggedright
Session 2 (Room B)
\end{minipage} & \begin{minipage}[b]{\linewidth}\raggedright
Session 3 (Room C)
\end{minipage} \\
\midrule
\endhead
July 15, Friday & \protect\hyperlink{oc1}{Optimal control 1} &
\protect\hyperlink{de1}{Differential equations 1} &
\protect\hyperlink{o1}{Optimization 1} \\
July 16, Saturday & \protect\hyperlink{meas}{Analysis and control in the
space of measures} & \protect\hyperlink{dea1}{Differential equations:
applications 1} & \protect\hyperlink{de2}{Differential equations 2} \\
July 17, Sunday & \protect\hyperlink{qc}{Quantum control} &
\protect\hyperlink{oc2}{Optimal control 2} &
\protect\hyperlink{de3}{Differential equations 3} \\
July 18, Monday & Trip to Baikal & Trip to Baikal & Trip to Baikal \\
July 19, Tuesday & & \protect\hyperlink{oc3}{Optimal control 3} &
\protect\hyperlink{dea2}{Differential equations: applications 2} \\
July 20, Wednesday & & \protect\hyperlink{dae}{Differential-algebraic
equations} & \protect\hyperlink{o2}{Optimization 2} \\
\bottomrule
\end{longtable}

\emph{Each session talk is 20 minutes long.}

\hypertarget{oc1}{%
\subsubsection{Optimal control 1}\label{oc1}}

\textbf{Chairs}: Alexander Strekalovsky, Dmitry Khlopin.

\begin{longtable}[]{@{}
  >{\raggedright\arraybackslash}p{(\columnwidth - 2\tabcolsep) * \real{0.0691}}
  >{\raggedright\arraybackslash}p{(\columnwidth - 2\tabcolsep) * \real{0.9309}}@{}}
\toprule
\begin{minipage}[b]{\linewidth}\raggedright
Time
\end{minipage} & \begin{minipage}[b]{\linewidth}\raggedright
Talk
\end{minipage} \\
\midrule
\endhead
16:30 & Alexander Strekalovsky (IDSCT SB RAS, Irkutsk, Russia). On
Nonconvex Optimal Control Problems. \\
16:50 & Dmitry Khlopin (IMM UrB RAS, Yekaterinburg, Russia). On
Necessary Conditions if Limits are Minimized. \\
17:10 & Evgeny Ladeyshchikov (Lomonosov Moscow State University, Moscow,
Russia), L. Lokutsievskiy. Time-optimal Problem on a Three-dimensional
Heisenberg Group. \\
17:30 & Ivan Osipov (IMM UrB RAS, Yekaterinburg, Russia). On the
Linearization Method in Small-time Control Synthesis. \\
17:50 & Vasilii Zaitsev, Inna Kim (UdSU, Izhevsk, Russia). On matrix
eigenvalue spectrum assignment for high-order linear systems by static
output feedback. \\
18:10 &
\protect\includegraphics[height=0.7em]{/tmp/Rtmp15D9Ng/file12e4364cb7303.pdf}
V.I. Berdyshev, Viktor B. Kostousov, A.A. Popov (IMM UrB RAS,
Yekaterinburg, Russia). Optimal Object Trajectories under Unfriendly
Observation. \\
\bottomrule
\end{longtable}

\hypertarget{oc2}{%
\subsubsection{Optimal control 2}\label{oc2}}

\textbf{Chairs}: Alexander Tyatyushkin, Alexander Y. Gornov.

\begin{longtable}[]{@{}
  >{\raggedright\arraybackslash}p{(\columnwidth - 2\tabcolsep) * \real{0.0542}}
  >{\raggedright\arraybackslash}p{(\columnwidth - 2\tabcolsep) * \real{0.9458}}@{}}
\toprule
\begin{minipage}[b]{\linewidth}\raggedright
Time
\end{minipage} & \begin{minipage}[b]{\linewidth}\raggedright
Talk
\end{minipage} \\
\midrule
\endhead
16:30 & Alexander Tyatyushkin (IDSCT SB RAS, Irkutsk, Russia). Control
optimization in systems with phase constraints. \\
16:50 & Alexander Y. Gornov, Tatyana Zarodnyuk (IDSCT SB RAS, Irkutsk,
Russia). The Modified Monowave Method for the Reachable Set Approximation
of the Nonlinear Controlled System on the Plane. \\
17:10 & Olga Samsonyuk (IDSCT SB RAS, Irkutsk, Russia). TBA. \\
17:30 &
\protect\includegraphics[height=0.7em]{/tmp/Rtmp15D9Ng/file12e4391014fd.pdf}
Alexey N. Rogalev (Institute of computing modelling SB RAS, Krasnoyarsk,
Russia). Numerical Estimation of the Boundaries of the Reachability Sets
of Controlled Systems Based on Symbolic Formulas. \\
17:50 &
\protect\includegraphics[height=0.7em]{/tmp/Rtmp15D9Ng/file12e4314597f0e.pdf}
Nyurgun Lazarev (North-Eastern Federal University, Yakutsk, Russia).
Optimal Location of Rigid Inclusions in Contact Problems for
Inhomogeneous Two-dimensional Bodies. \\
18:10 &
\protect\includegraphics[height=0.7em]{/tmp/Rtmp15D9Ng/file12e4314ef8024.pdf}
Igor' Izmest'ev (IMM UrB RAS, Yekaterinburg, Russia). Grid Algorithm for
Computing Reachability Sets with a Modified Reduction Procedure. \\
\bottomrule
\end{longtable}

\hypertarget{oc3}{%
\subsubsection{Optimal control 3}\label{oc3}}

\textbf{Chairs}: Vladimir Dykhta, Stepan Sorokin.

\begin{longtable}[]{@{}
  >{\raggedright\arraybackslash}p{(\columnwidth - 2\tabcolsep) * \real{0.0442}}
  >{\raggedright\arraybackslash}p{(\columnwidth - 2\tabcolsep) * \real{0.9558}}@{}}
\toprule
\begin{minipage}[b]{\linewidth}\raggedright
Time
\end{minipage} & \begin{minipage}[b]{\linewidth}\raggedright
Talk
\end{minipage} \\
\midrule
\endhead
16:30 & Vladimir Dykhta (IDSCT SB RAS, Irkutsk, Russia). Feedback
minimum principle: variational strenthening of the concept of
extremality in optimal control. \\
16:50 & Alexander Arguchintsev, Vasilisa Poplevko (Institute of
Mathematics and Information Technologies, Irkutsk State University,
Irkutsk, Russia). Variational Optimality Condition in Control of
Hyperbolic Systems with Boundary Delay Parameters. \\
17:10 & Boris Ananyev, Polina Yurovskikh (IMM UrB RAS, Yekaterinburg,
Russia). Estimation Problem for Discrete Systems with Information
Delays. \\
17:30 &
\protect\includegraphics[height=0.7em]{/tmp/Rtmp15D9Ng/file12e432ede7cfd.pdf}Nina
N. Subbotina, Evgenii F. Krupennikov (IMM UrB RAS, UrFU, Ekaterinburg,
Russia). Stationary points of d.c. Lagrangians in solving inverse
problems of the control theory. \\
17:50 &
\protect\includegraphics[height=0.7em]{/tmp/Rtmp15D9Ng/file12e43239de981.pdf}
Lyubov Shagalova (IMM UrB RAS, Yekaterinburg, Russia). On the Solution
of the Hamilton-Jacobi Equation with State Constraints Given by Zeros of
the Coefficients at the Exponential Terms of the Hamiltonian. \\
18:10 &
\protect\includegraphics[height=0.7em]{/tmp/Rtmp15D9Ng/file12e431fe1ce03.pdf}
Ilya Chupin, Yurii Dolgii (Ural Federal university, Ekaterinburg,
Russia). Optimal Control of Manipulator. \\
18:30 &
\protect\includegraphics[height=0.7em]{/tmp/Rtmp15D9Ng/file12e43538d333c.pdf}
Vladimir A. Dubovitskij (Institute of Problems of Chemical Physics,
Chernogolovka, Russia). On the Right Invertibility of the Differential
for the Equality Constraint Operator and the Implicit Function Theorem
in a General Optimal Control Problem. \\
\bottomrule
\end{longtable}

\hypertarget{de1}{%
\subsubsection{Differential equations 1}\label{de1}}

\textbf{Chairs}: Inessa Matveeva, Valery Gaiko.

\begin{longtable}[]{@{}
  >{\raggedright\arraybackslash}p{(\columnwidth - 2\tabcolsep) * \real{0.0476}}
  >{\raggedright\arraybackslash}p{(\columnwidth - 2\tabcolsep) * \real{0.9524}}@{}}
\toprule
\begin{minipage}[b]{\linewidth}\raggedright
Time
\end{minipage} & \begin{minipage}[b]{\linewidth}\raggedright
Talk
\end{minipage} \\
\midrule
\endhead
16:30 & Inessa Matveeva (Sobolev Institute of Mathematics SB RAS,
Novosibirsk, Russia). Estimates for Solutions to Some Classes of
Nonautonomous Nonlinear Time-delay Systems. \\
16:50 & Valery Gaiko (United Institute of Informatics Problems, National
Academy of Sciences of Belarus). Catastrophe Theory and Global
Bifurcations of Limit Cycles. \\
17:10 & Elena Chistyakova (IDSCT SB RAS, Irkutsk, Russia). Solving a
Heat Mass Transfer Problem Using Differential Algebraic Equations. \\
17:30 & Andrey Muravnik (S.M. Nikol'skii Mathematical Institute of RUDN,
Moscow, Russia). Qualitative theory of equations and inequalities with
KPZ-nonlinearities. \\
17:50 &
\protect\includegraphics[height=0.7em]{/tmp/Rtmp15D9Ng/file12e4360397a92.pdf}
Vyacheslav V. Provotorov (Voronezh State University, Voronezh, Russia),
Semen L. Podvalny (Voronezh State Technical University, Voronezh,
Russia). Navier-Stokes Evolutionary System with Spatial Variable in a
Network-like Domain. \\
18:10 &
\protect\includegraphics[height=0.7em]{/tmp/Rtmp15D9Ng/file12e4314a61a2c.pdf}
Maxim V. Shamolin (Lomonosov Moscow State University, Moscow, Russia).
Tensor Invariants of Dynamical Systems with Dissipation. \\
\bottomrule
\end{longtable}

\hypertarget{de2}{%
\subsubsection{Differential equations 2}\label{de2}}

\textbf{Chairs}: Alexander Kosov, Ivan A. Finogenko.

\begin{longtable}[]{@{}
  >{\raggedright\arraybackslash}p{(\columnwidth - 2\tabcolsep) * \real{0.0535}}
  >{\raggedright\arraybackslash}p{(\columnwidth - 2\tabcolsep) * \real{0.9465}}@{}}
\toprule
\begin{minipage}[b]{\linewidth}\raggedright
Time
\end{minipage} & \begin{minipage}[b]{\linewidth}\raggedright
Talk
\end{minipage} \\
\midrule
\endhead
16:30 & Alexander Kosov, Edward Semenov (IDSCT SB RAS, Irkutsk, Russia).
On Exact Solutions of Equations Used in Modeling the Motion of
Distributed Formations. \\
16:50 & Ivan A. Finogenko (IDSCT SB RAS, Irkutsk, Russia). Method of
Limiting Differential Inclusions for Discontinuous Systems. \\
17:10 & Timur Yskak (Sobolev Institute of Mathematics SB RAS,
Novosibirsk, Russia). About Exponential Stability of Solutions to
Systems of Differential Equations of Neutral Type with Distributed
Delay. \\
17:30 & Margarita V. Artemeva, M.O. Korpusov. (Lomonosov Moscow State
University, Moscow, Russia). Blow up of Solutions and Local Solvability
of an Abstract Cauchy Problem for Second-order Differential Equation
with a Non-coercive Source. \\
17:50 & Nikolay Sidorov, Lev Sidorov (ISU, Irkutsk, Russia). On the
Spectrum of One Class of Integral-Functional Operators in Solving
Nonlinear Volterra Loaded Equations. \\
18:10 &
\protect\includegraphics[height=0.7em]{/tmp/Rtmp15D9Ng/file12e434723689c.pdf}
Andrey L. Ushakov (South Ural State University (National Research
University), Chelyabinsk, Russia). Nonlinear Analysis Mixed Boundary
Value Problem for the Sophie Germain Equation. \\
\bottomrule
\end{longtable}

\hypertarget{de3}{%
\subsubsection{Differential equations 3}\label{de3}}

\textbf{Chair}: Anna Lempert.

\begin{longtable}[]{@{}
  >{\raggedright\arraybackslash}p{(\columnwidth - 2\tabcolsep) * \real{0.0401}}
  >{\raggedright\arraybackslash}p{(\columnwidth - 2\tabcolsep) * \real{0.9599}}@{}}
\toprule
\begin{minipage}[b]{\linewidth}\raggedright
Time
\end{minipage} & \begin{minipage}[b]{\linewidth}\raggedright
Talk
\end{minipage} \\
\midrule
\endhead
16:30 &
\protect\includegraphics[height=0.7em]{/tmp/Rtmp15D9Ng/file12e4377392dec.pdf}
Lina Bondar (IM SO RAN, Novosibirsk, Russia), Sanzhar Mingnarov (NSU,
Novosibirsk, Russia). On solvability of the Cauchy problem for one
pseudohyperbolic system. \\
16:50 &
\protect\includegraphics[height=0.7em]{/tmp/Rtmp15D9Ng/file12e431a0e1c2.pdf}
Nikita O. Ivanov (RUDN University, Moscow, Russia). On Generalized
Solutions of the Second Boundary Value Problem for
Differential-difference Equations with Variable Coefficients. \\
17:10 &
\protect\includegraphics[height=0.7em]{/tmp/Rtmp15D9Ng/file12e4372c2de6d.pdf}
V. Obukhovskii, Garik Petrosyan, M. Soroka (Voronezh State Pedagogical
University, Voronezh State University of Engineering Technologies,
Voronezh, Russia). On the Solvability of a Nonlocal Boundary Value
Problem for Fractional Differential Inclusions with Causal
Multioperators. \\
17:30 &
\protect\includegraphics[height=0.7em]{/tmp/Rtmp15D9Ng/file12e43424feba1.pdf}
Ekaterina I. Zotova, R.D. Murtazina (USATU, Ufa, Russia). Laplace
Cascade method. \\
17:50 &
\protect\includegraphics[height=0.7em]{/tmp/Rtmp15D9Ng/file12e43634e1bba.pdf}
Andrey Osipov (Federal State Institution ``Scientific-Research Institute
for System Analysis of the Russian Academy of Sciences'', Moscow,
Russia). On an inverse spectral problem for band operators and nonlinear
lattices. \\
18:10 &
\protect\includegraphics[height=0.7em]{/tmp/Rtmp15D9Ng/file12e433a7ece66.pdf}
Anatoly Aristov (MSU, Moscow, Russia). Exact Solutions of a Nonclassical
Nonlinear Partial Differential Equation. \\
\bottomrule
\end{longtable}

\hypertarget{dea1}{%
\subsubsection{Differential equations: applications 1}\label{dea1}}

\textbf{Chair}: Alexander Kazakov

\begin{longtable}[]{@{}
  >{\raggedright\arraybackslash}p{(\columnwidth - 2\tabcolsep) * \real{0.0528}}
  >{\raggedright\arraybackslash}p{(\columnwidth - 2\tabcolsep) * \real{0.9472}}@{}}
\toprule
\begin{minipage}[b]{\linewidth}\raggedright
Time
\end{minipage} & \begin{minipage}[b]{\linewidth}\raggedright
Talk
\end{minipage} \\
\midrule
\endhead
16:30 & Maria Skvortsova (Sobolev Institute of Mathematics SB RAS,
Novosibirsk, Russia). On a model of population dynamics with several
delays. \\
16:50 & Evgeny Rudoy (Sobolev Institute of Mathematics of SB RAS,
Lavrentyev Institute of Hydrodynamics SB RAS, Novosibirsk, Russia).
Asymptotic Modeling of Interfaces in Kirchhoff-Love's Plates Theory. \\
17:10 & Pavel Kuznetsov, Alexander Kazakov (IDSCT SB RAS, Irkutsk,
Russia). On Analytical Solvability of the Problem with a Given Zero
Front for the Nonlinear Parabolic Predator-Prey System. \\
17:30 &
\protect\includegraphics[height=0.7em]{/tmp/Rtmp15D9Ng/file12e433a37bcaf.pdf}
Yulia O. Koroleva (Gubkin State University of Oil and Gas, HSE, Moscow,
Russia). On the Weak Solution of the Electro-Hydrodynamical Boundary
Value Problem for the Unit Cell of Cation-exchange Membrane. \\
17:50 &
\protect\includegraphics[height=0.7em]{/tmp/Rtmp15D9Ng/file12e431a241635.pdf}
Mariia I. Delova, Olga S. Rozanova (Lomonosov Moscow State University,
Moscow, Russia). On Multidimensional Oscillations of a Cold Plasma with
Account for Electron-ion Collisions. \\
18:10 &
\protect\includegraphics[height=0.7em]{/tmp/Rtmp15D9Ng/file12e433e942abf.pdf}
Viktor Korzyuk, Jan Rudzko (Belarusian State University, Minsk,
Belarus). Classical Solution of the First Mixed Problem for the
Telegraph Equation with a Nonlinear Potential. \\
\bottomrule
\end{longtable}

\hypertarget{dea2}{%
\subsubsection{Differential equations: applications 2}\label{dea2}}

\textbf{Chair}: Elena Chistyakova

\begin{longtable}[]{@{}
  >{\raggedright\arraybackslash}p{(\columnwidth - 2\tabcolsep) * \real{0.0504}}
  >{\raggedright\arraybackslash}p{(\columnwidth - 2\tabcolsep) * \real{0.9496}}@{}}
\toprule
\begin{minipage}[b]{\linewidth}\raggedright
Time
\end{minipage} & \begin{minipage}[b]{\linewidth}\raggedright
Talk
\end{minipage} \\
\midrule
\endhead
16:30 &
\protect\includegraphics[height=0.7em]{/tmp/Rtmp15D9Ng/file12e43e58b427.pdf}
Tamara G. Sukacheva (The Yaroslav-the-Wise Novgorod State University
(NovSU), Veliky Novgorod, Russia). Oskolkov models and Sobolev-type
equations in magnetohydrodynamics. \\
16:50 &
\protect\includegraphics[height=0.7em]{/tmp/Rtmp15D9Ng/file12e4369ec2983.pdf}
Aleksey O. Kondyukov (The Yaroslav-the-Wise Novgorod State University
(NovSU), Veliky Novgorod, Russia). The First Initial-boundary Value
Problem for Oskolkov System of Nonzero Order. \\
17:10 &
\protect\includegraphics[height=0.7em]{/tmp/Rtmp15D9Ng/file12e4372d5c43e.pdf}
Alexandre Demidov (MSU, Moscow, Russia). Planar Flows with Minimal Ratio
of the Extremal Values of the Pressure on the Free Boundary. \\
17:30 &
\protect\includegraphics[height=0.7em]{/tmp/Rtmp15D9Ng/file12e4324a5d025.pdf}
Aigul A. Mukhutdinova (Mavlyutov Institute of Mechanics, Ufa
Investigation Center, RAS, Russia), A.D. Nizamova, V.N. Kireev, S.F.
Urmancheev. Spectral Analysis of the Stability of Fluid Flow in an
Annular Channel. \\
17:50 &
\protect\includegraphics[height=0.7em]{/tmp/Rtmp15D9Ng/file12e43454f790d.pdf}
Eric R. Shaihiev (USATU, Ufa, Russia), A.D. Nizamova (Mavlyutov
Institute of Mechanics, Ufa Investigation Center, RAS, Russia). Fluid
Storage Control with a Proportional-integrally Differentiating
Solver. \\
18:10 &
\protect\includegraphics[height=0.7em]{/tmp/Rtmp15D9Ng/file12e4360afe3da.pdf}
Karuppaiya Sakkaravarthi (Asia-Pacific Center for Theoretical Physics
(APCTP), Republic of Korea). Bright Solitons in a (2+1)-dimensional
Oceanic Model: Dynamics, Interaction and Molecule Formation. \\
\bottomrule
\end{longtable}

\hypertarget{dae}{%
\subsubsection{Differential-algebraic equations}\label{dae}}

\textbf{Chairs}: Alla A. Shcheglova, Svetlana Svinina.

\begin{longtable}[]{@{}
  >{\raggedright\arraybackslash}p{(\columnwidth - 2\tabcolsep) * \real{0.0514}}
  >{\raggedright\arraybackslash}p{(\columnwidth - 2\tabcolsep) * \real{0.9486}}@{}}
\toprule
\begin{minipage}[b]{\linewidth}\raggedright
Time
\end{minipage} & \begin{minipage}[b]{\linewidth}\raggedright
Talk
\end{minipage} \\
\midrule
\endhead
15:10 & Ekaterina Antipina (Melentiev Energy Systems Institute SB RAS,
Irkutsk, Russia), Mikhail Bulatov (IDSCT SB RAS, Irkutsk, Russia),
Vitaly Biryukov. Block Integral Methods for the Numerical Solution of
the Volterra Equation of the First Kind. \\
15:30 & Svetlana Svinina (IDSCT SB RAS, Irkutsk, Russia). On the
Numerical Solution of Linear Multidimensional Differential-algebraic
Systems. \\
15:50 & Liubov Solovarova (ISDCT SB RAS, Irkutsk, Russia), Ta Duy
Phuong. On numerical solution of the second order differential-algebraic
equations. \\
16:10 & Alla A. Shcheglova. Impulse Response Matrix for Time-Varying
System of Differential-Algebraic Equations. \\
16:30 & Pavel Petrenko (IDSCT SB RAS, Irkutsk, Russia). A Note on
Differential-algebraic Equations with Hysteresis Phenomena. \\
16:50 & E. Yu. Grazhdantseva, Svetlana V. Solodusha (Irkutsk State
University, Melentiev Energy Systems Institute SB RAS, Irkutsk, Russia).
On an Analytical Solution of a Nonlinear Partial Differential
Equation. \\
\bottomrule
\end{longtable}

\hypertarget{o1}{%
\subsubsection{Optimization 1}\label{o1}}

\textbf{Chairs}: Bazaragchaa Barsbold, Viktor F. Chistyakov.

\begin{longtable}[]{@{}
  >{\raggedright\arraybackslash}p{(\columnwidth - 2\tabcolsep) * \real{0.0483}}
  >{\raggedright\arraybackslash}p{(\columnwidth - 2\tabcolsep) * \real{0.9517}}@{}}
\toprule
\begin{minipage}[b]{\linewidth}\raggedright
Time
\end{minipage} & \begin{minipage}[b]{\linewidth}\raggedright
Talk
\end{minipage} \\
\midrule
\endhead
16:30 & Bazaragchaa Barsbold (School of Engineering and Applied
Sciences, National University of Mongolia, Ulaanbaatar, Mongolia),
Balkhuu Batbayasgalan, Dovdon Batsuuri, Dorjkhuu Enkhtaivan. A
Sequential Approach to a Minimum Norm Partial Pole Assignment
Problem. \\
16:50 & Viktor F. Chistyakov (IDSCT SB RAS, Irkutsk, Russia). On the
Reduction of a Singular Linear-quadratic Control Problem to the Problem
of Calculus of Variations. \\
17:10 & Anton Anikin (IDSCT SB RAS, Irkutsk, Russia). About One
Modification of Broyden-family Quasi-Newton Methods. \\
17:30 & Pavel Sorokovikov (IDSCT SB RAS, Irkutsk, Russia). Combined
algorithms based on bioinspired and local search methods for solving
multiextremal optimization problems. \\
17:50 & Vsevolod Voronov, Viktoria Svistunova (Caucasus Mathematical
Center of ASU, Maikop, Russia). Optimization of sphere partitions and
estimates of the chromatic number for a forbidden interval of
distances. \\
18:10 &
\protect\includegraphics[height=0.7em]{/tmp/Rtmp15D9Ng/file12e43600f163.pdf}
Sergey Kabanikhin (Novosibirsk State University, Novosibirsk, Russia).
Nonlinear Inverse Problems and Optimization. \\
\bottomrule
\end{longtable}

\hypertarget{o2}{%
\subsubsection{Optimization 2}\label{o2}}

\textbf{Chair}: Alexander Gornov.

\begin{longtable}[]{@{}
  >{\raggedright\arraybackslash}p{(\columnwidth - 2\tabcolsep) * \real{0.0528}}
  >{\raggedright\arraybackslash}p{(\columnwidth - 2\tabcolsep) * \real{0.9472}}@{}}
\toprule
\begin{minipage}[b]{\linewidth}\raggedright
Time
\end{minipage} & \begin{minipage}[b]{\linewidth}\raggedright
Talk
\end{minipage} \\
\midrule
\endhead
15:10 &
\protect\includegraphics[height=0.7em]{/tmp/Rtmp15D9Ng/file12e43725d7db2.pdf}
Valentin Gorokhovik (Institute of Mathematics of the National Academy of
Sciences of Belarus, Minsk, Belarus). Separation of Convex Sets by
Halfspaces with Applications to Convex Optimization Problems. \\
15:30 &
\protect\includegraphics[height=0.7em]{/tmp/Rtmp15D9Ng/file12e43bb3e76.pdf}
Igor Zabotin, Oksana Shulgina, Rashid Yarullin (Kazan (Volga region)
Federal University, Kazan, Russia). One variant of the Two-Stage
Cutting-Plane Method. \\
15:50 &
\protect\includegraphics[height=0.7em]{/tmp/Rtmp15D9Ng/file12e435488a408.pdf}
I.Ya. Zabotin, Kseniya E. Kazaeva, O.N. Shulgina (Kazan (Volga Region)
Federal University, Kazan, Russia). Variant of the Objective Function
Parametrization Method for a Convex Programming Problem. \\
16:10 &
\protect\includegraphics[height=0.7em]{/tmp/Rtmp15D9Ng/file12e43242dc8e9.pdf}
Igor Prudnikov (Scientific Center of Smolensk State Medical University,
Smolensk, Russia). Constructions of the subdifferentials and
codifferentials. \\
16:30 &
\protect\includegraphics[height=0.7em]{/tmp/Rtmp15D9Ng/file12e436586b179.pdf}
Vadim Zizov (Lomonosov Moscow State University, Moscow, Russia). Lower
Bounds for Area Complexity of Decoder in Model of Cellular Circuits. \\
16:50 &
\protect\includegraphics[height=0.7em]{/tmp/Rtmp15D9Ng/file12e435d98b905.pdf}
Akmal Mamatov, Islom Ravshanov (Samarkand state university, Samarkand,
Uzbekistan). Algorithm for solving one maximin problem with connected
variables. \\
17:10 &
\protect\includegraphics[height=0.7em]{/tmp/Rtmp15D9Ng/file12e43388747f7.pdf}
Akmal Mamatov (Samarkand state university, Samarkand, Uzbekistan). On
the Theory of Game Problems with Connected Variables. \\
\bottomrule
\end{longtable}

\hypertarget{meas}{%
\subsubsection{Analysis and control in the space of
measures}\label{meas}}

\textbf{Chairs}: Dmitrii Serkov, Dmitry Khlopin.

\begin{longtable}[]{@{}
  >{\raggedright\arraybackslash}p{(\columnwidth - 2\tabcolsep) * \real{0.0591}}
  >{\raggedright\arraybackslash}p{(\columnwidth - 2\tabcolsep) * \real{0.9409}}@{}}
\toprule
\begin{minipage}[b]{\linewidth}\raggedright
Time
\end{minipage} & \begin{minipage}[b]{\linewidth}\raggedright
Talk
\end{minipage} \\
\midrule
\endhead
16:30 & Dmitrii Serkov, Alexander Chentsov (IMM UrB RAS, Yekaterinburg,
Russia). On a property of continuous dependence of sets in the space of
measures. \\
16:50 & Yurii Averboukh, Dmitry Khlopin (IMM UrB RAS, Yekaterinburg,
Russia). Necessary optimality condition for deterministic mean field
type control problem. \\
17:10 & Olga Yufereva (IMM UrB RAS, Yekaterinburg, Russia), Michael
Persiianov, Pavel Dvurechensky, Alexander Gasnikov, Dmitry Kovalev.
Decentralized Computation of Wasserstein Barycenter over Time-Varying
Networks. \\
17:30 & Nikolay Podogaev (IDSCT SB RAS, Irkutsk, Russia), Maxim
Staritsyn. Numerical solution of optimal control problems for nonlocal
continuity equations. \\
17:50 & Nikolay Pogodaev, Maxim Staritsyn (IDSCT SB RAS, Irkutsk,
Russia), Fernando Lobo Pereira. Exact increment formulas for optimal
control in the space of probability measures. \\
\bottomrule
\end{longtable}

\hypertarget{qc}{%
\subsubsection{Quantum control}\label{qc}}

\textbf{Chair}: Oleg V. Morzhin.

\begin{longtable}[]{@{}
  >{\raggedright\arraybackslash}p{(\columnwidth - 2\tabcolsep) * \real{0.0464}}
  >{\raggedright\arraybackslash}p{(\columnwidth - 2\tabcolsep) * \real{0.9536}}@{}}
\toprule
\begin{minipage}[b]{\linewidth}\raggedright
Time
\end{minipage} & \begin{minipage}[b]{\linewidth}\raggedright
Talk
\end{minipage} \\
\midrule
\endhead
16:30 & Boris Volkov (Steklov Mathematical Institute RAS, Moscow,
Russia), Alexander Pechen. Traps in quantum control landscapes. \\
16:50 & Anastasia A. Myachkova (Steklov Mathematical Institute RAS,
Moscow, Russia), Alexander N. Pechen. Analysis of the controllability
criteria for some degenerate four-level quantum systems. \\
17:10 & Sergey Kuznetsov, Alexander Pechen (Steklov Mathematical
Institute RAS, Moscow, Russia). On Controllability of a Highly
Degenerate Four-level Quantum System with a «Chained» Coupling
Hamiltonian. \\
17:30 &
\protect\includegraphics[height=0.7em]{/tmp/Rtmp15D9Ng/file12e437a76319d.pdf}
Oleg V. Morzhin (Steklov Mathematical Institute RAS, National University
of Science and Technology ``MISiS'', Moscow, Russia). On Optimizing
Coherent and Incoherent Controls in Some Open Quantum Systems. \\
17:50 &
\protect\includegraphics[height=0.7em]{/tmp/Rtmp15D9Ng/file12e43c773602.pdf}
Vadim Petruhanov (Steklov Mathematical Institute RAS, Moscow, Russia,
Moscow Institute of Physics and Technology, Dolgoprudny, Russia),
Alexander Pechen. GRAPE Method for Open Quantum Systems Driven by
Coherent and Incoherent Controls. \\
\bottomrule
\end{longtable}

\end{document}
